In dit hoofdstuk staan de resultaten van het onderzoek per deelvraag uitgewerkt. 
Er is voornamelijk literatuur- en component specifieke documentatieonderzoek als methode gebruikt.  
%Kwalitatief onderzoek is beschreven met behulp van proza. De metingen van kwantitatief onderzoek zijn verwerkt tot grafieken.
%Observaties en resultaten van literatuuronderzoek zijn met proza beschreven.
\subsection{Op welke simpele manier kunnen de meetgegevens van de DHT11-sensor realtime uitgelezen worden door een ESP32 MCU?}
\label{simpel}
\begin{itemize}
    \item[\ding{226}] Onderzoek van de technische specificaties van de DHT11-sensor
\end{itemize}
Volgens de datasheet \cite{dht11} van de DHT11-sensor zijn de onderstaande gegevens belangrijk als uitgangspunt: 
\begin{table}[h]
    \centering
    \label{table:tabel12}
    \begin{tabular}{|l|l|}
    \hline
    \textbf{Specificaties}               & \textbf{DHT11-11 sensor}        \\ \hline
    \textit{Communicatie Protocol}       & \textit{Single-Wire Two-Way} \\ \hline
    \textit{Temperatuur meetbereik}      & 0 - 50 $^\circ$C      \\ \hline
    \textit{Luchtvochtigheid meetbereik} & 20 - 90\% RH      \\ \hline
    \textit{Temperatuur nauwkeurigheid}      & $\pm 2^\circ$C      \\ \hline
    \textit{Luchtvochtigheid nauwkeurigheid} & $\pm 5$\% RH      \\ \hline
    \textit{Voedingsbereik}              & 3 - 5.5V DC                     \\ \hline
    \textit{``Sample period''}           & 2 secondes                    \\ \hline
    \end{tabular}
    \linebreak
    \caption{Technische specificaties van de DHT11-sensor.}
\end{table}\\
Deze sensor bevat een \textit{resistive-type} luchtvochtigheidsmetingscomponent en een NTC-component voor de temperatuurmeting. 
De sensor heeft in totaal vier pennen, namelijk:
\begin{itemize}
    \item Pen-1: De $V_{cc}$ ingang met een spanningsingang van 3.3V DC;
    \item Pen-2: De data pen. Deze kan worden aangesloten aan een van de GPIO (\textit{General Purpose Input Output}) pennen van een MCU;
    \item Pen-3: Deze pen wordt niet gebruikt;
    \item Pen-4: De $GND$ pen. Dit is de grond-pen van de DHT11-sensor. 
\end{itemize}
De datasheet raadt aan om een 5$k\Omega$ \textit{pull-up resistor} te gebruiken wanneer de kabellengte langer is dan 20 meter.

Als tweede stap is ook een literatuuronderzoek gedaan naar de werking van de DHT11-sensor. 
Uit deze literatuuronderzoek \cite{G2018} is gebleken dat de DHT11-sensor een \textit{slave device} is. 
De DHT11-sensor stuurt pas gegevens wanneer de sensor wordt aangespoord door zijn \textit{master device}. 
De MCU (in dit geval de \textit{master}) doet een verzoek op de databus en wacht tot dat de sensor reageert met meetdata. 
De DHT11-sensor zal dan vervolgens de gemeten data terug koppelen aan de MCU. 
Voor het uitlezen van data door de MCU zal deze wijze toegepast moeten worden bij het programmeren. 
Voor het programmeren kunnen C of C++ als taal gebruikt worden. 
\begin{itemize}
    \item[\ding{226}] Onderzoek van de technische specificaties van de ESP32 MCU
\end{itemize}
De datasheet \cite{espressif} van de ESP32 MCU is onderzocht. In \autoref{fig:pinout} is de \textit{pinout}-schema van deze MCU weergegeven. 
Hieronder is een lijst weergegeven met de essentiële technische specificaties die relevant zijn voor het beantwoorden van deze deelvraag.
\begin{itemize}
    \item Wi-Fi: IEEE 802.11b/g/n, hierbij wordt alleen 2.4 GHz ondersteund;
    \item Dual-core 32-bit LX6 microprocessor;
    \item 448 KB ROM en 520 KB SRAM; 
    \item 34 GPIO (\textit{General Purpose Input Output}) pennen;
    \item Arduino (C++) \textit{compatible}.
\end{itemize}
Als tweede stap is ook literatuuronderzoek \cite{Maier2017} gedaan naar de werking van de ESP32 MCU. 
Deze MCU draait een \textit{realtime} operating system genaamd FreeRTOS. 
Het is een \textit{dual-core} systeem dat ontworpen is om de volgende protocollen te kunnen gebruiken:
\begin{itemize}
    \item TCP/IP;
    \item 802.11 WLAN;
    \item MAC.
\end{itemize}
Deze MCU heeft 34 GPIO-pennen, en met deze pennen kunnen \textit{slave}- en/of \textit{master devices} op aangesloten worden. 
De ESP32 wordt geprogrammeerd in C en in C++ (Arduino). 
% Het bedrijf achter de ESP32 MCU heeft ook een documentatie geschreven over de programmeermogelijkheden, deze heet de ``\textit{Espressif Systems Internet of Things development framework}''. 
Om de ESP32 te programmeren in C++ zijn er ook Arduino bibliotheken beschikbaar. 
\begin{itemize}
    \item[\ding{226}] Literatuuronderzoek naar het uitlezen van meetgegevens door een MCU
\end{itemize}
Om de meetgegevens van de DHT11-sensor af te lezen, wordt gebruik gemaakt van programmeertaal C++, hierbij wordt de \verb|<DHT11.h>| Arduino bibliotheek gebruikt \cite{Cameron2019}. 
De pennen dienen op de juiste manier aangesloten te worden, zoals eerder is aangegeven. 
Zie \autoref{fig:ping} voor de aansluitingsschema en \autoref{fig:sch} voor de schematische weergave ervan.

Nadat in het C++ code de \verb|<DHT11.h>| Arduino bibliotheek is toegevoegd kan het DHT11-object worden aangemaakt. 
Tijdens het definiëren van het object moeten in de parameters aangegeven worden op welke GPIO-pen de sensor aangesloten is, en welke type DHT11-sensor het is, zie hieronder: 
\begin{lstlisting}[language=C++]
DHT11 dht_sensor(21, DHT11);
\end{lstlisting}
Om de data van de DHT11-sensor op een juiste manier af te kunnen lezen dienen variabelen aangemaakt te worden voor de luchtvochtigheid en temperatuur. 
Voor het opslaan van de gemeten data in deze variabelen, worden deze als volgt toegewezen:
\begin{lstlisting}[language=C++]
float luchtVocht  = dht_sensor.readHumidity();
float tempGC = dht_sensor.readTemperature();
\end{lstlisting}

Om de juiste werking van DHT11-sensor te kunnen testen kan er eventueel gebruik gemaakt worden van de print functie \cite{6997578}, zie hieronder:
\begin{lstlisting}[mathescape, language=C++]
Serial.print("Luchtvochtigheid: ");
Serial.print(luchtVocht);
Serial.print("% RH");
Serial.print(" en ");
Serial.print("Temperatuur: ");
Serial.print(tempGC);
Serial.print("${^\circ}$C");
\end{lstlisting}

% \subsection{Op welke manier kan een \textit{stand-alone} Wi-Fi netwerk worden geprogrammeerd op de ESP32 MCU?}
% \subsection{Op welke manier kan een simpele \textit{access point} worden geprogrammeerd op de ESP32 MCU?}
\subsection{Verken op welke elementaire wijze een Wi-Fi netwerk opgezet kan worden met de ESP32 MCU?}
\begin{itemize}
    \item[\ding{226}] Bestuderen van ESP-IDF documentatie ter verkennen van Wi-Fi mogelijkheden
\end{itemize}
Om erachter te komen hoe op een simpele wijze een Wi-Fi netwerk te programmeren is met de ESP32 MCU, is eerst 
een documentatieonderzoek gedaan (\textit{deskresearch}) naar welke manieren er beschikbaar zijn om Wi-Fi te configureren op de ESP32 MCU. 
Volgens de officiële \textit{ESPressif-IDF documentation} \cite{unknowns} zijn er in totaal twee methoden beschikbaar om een Wi-Fi netwerk te configureren op de ESP32 MCU, te weten:  
\begin{enumerate}
    \item De \textit{Access Point}-modus: Hiermee kan een AP (\textit{access point}) geconfigureerd worden, waarna andere apparaten (die Wi-Fi ondersteunen) kunnen verbinden met de ESP32 MCU;
    \item STA/AP-modus: Hiermee kan de ESP32 MCU zowel een client zijn van een andere AP, als een eigen AP terggelijk zijn.
\end{enumerate}
Daarnaast zijn er ook nog andere configuratie mogelijkheden beschikbaar voor de AP, STA en STA/AP-modi, namelijk:
\begin{enumerate}
    \item Verschillende beveiligingsmodi: denk aan beveiligingen zoals WPA, WPA2, WEP;
    \item Continu-scan modus: De ESP32 MCU zal op een actieve wijze scannen naar toegangspunten;
    \item Promiscue modus: Deze modus wordt gebruikt voor het monitoren en bijhouden van IEEE 802.11 (WLAN) pakketten.
  \end{enumerate}
\begin{itemize}
    \item[\ding{226}] Vergelijk de verschillende WiFi methoden en configuratie mogelijkheden van ESP32 MCU om daarna een juiste keuze te kunnen maken voor een elementaire Wi-Fi netwerk opzet
\end{itemize}
Een elementaire Wi-Fi netwerk moet een los Wi-FI netwerk zijn, daarom valt de STA/AP af. 
Daarnaast moet dit netwerk simpel zijn en zijn de continu-scan- en promiscue modi niet van belang. 
Tenslotte is er gekozen om geen beveiligingsmodi te configureren vanwege elementaire opzet voor een Wi-FI netwerk. 
De aangegeven vergelijking heeft geresulteerd tot het toepassen van de AP methode, zonder toevoeging van extra configuratie mogelijkheden.
% In dit artikel is gekozen om een Wi-Fi netwerk te maken door gebruik van \verb|SoftAP| (naam van de AP methode in Arduino). 
\begin{itemize}
    \item[\ding{226}] Realiseer met de gekozen Wi-Fi netwerk mogelijkheid de benodigde elementaire Wi-Fi netwerk
\end{itemize}
Het configureren van de AP op de ESP32 MCU wordt gerealiseerd door de ``WiFi.h'' bibliotheek te gebruiken. 
In deze bibliotheek wordt de AP functionaliteit ook wel \textit{SoftAP} genoemd. 
Daarnaast moet de AP een SSID (\textit{Service Set IDentifier}, A.K.A. netwerk naam) hebben en kan dit met behulp van de volgende code toegepast worden \cite{Charles2013}:
\begin{lstlisting}[language=C++]
#include <WiFi.h>
/* Naam van netwerk: */
const char* ssid = "DHT11-sensor-netwerk"; 
/* (Geen) Wachtwoord van netwerk: */
const char* password = NULL;
\end{lstlisting}
De AP moet ook gekoppeld worden aan een IP-adres, zodat een gebruiker verbinding kan maken met de AP. 
Met deze onderstaande configuratie wordt het IP-adres, de default gateway en het subnet-masker van de AP geconfigureerd \cite{upesy}:
\begin{lstlisting}[language=C++]
IPAddress local_ip(192, 168, 1, 1);
IPAddress gateway(192, 168, 1, 1);
IPAddress subnet(255, 255, 255, 0);
\end{lstlisting}
Om nu de AP te starten hoeft alleen nog in de \verb|void setup()| functie de \verb|WiFi.SoftAP()| en \verb|WiFi.softAPConfig| methoden opgeroepen te worden:
\begin{lstlisting}[language=C++]
void setup() {
    Serial.begin(115200);
    WiFi.softAP(ssid, password);
    WiFi.softAPConfig(local_ip, gateway, subnet);
  }
\end{lstlisting}
\subsection{Op welke manier kunnen de meetgegevens continu ververst en gepresenteerd worden via de webinterface?}
\label{MEET}
\begin{itemize}
    \item[\ding{226}] Literatuuronderzoek naar mogelijke programmeermogelijkheden die gebruikt kunnen worden om een webpagina te verversen
\end{itemize}
Volgens de uitgevoerde literatuuronderzoek \cite{Lau2020} zijn er meerdere mogelijkheden om een webpagina te verversen. 
Een van de mogelijkheden is het gebruik van JS (\textit{JavaScript}) code. 
De JS code gebruikt de \verb|location.refresh()| methode om de webpagina te verversen.
De JS code kan worden geïmplementeerd in de \textit{source code} van de HTML webpagina. 

Een andere mogelijkheid om de webpagina te verversen is het gebruik van \verb|<meta>| binnen in de HTML code. 
De \verb|<meta>| tag staat in de header van de HTML code en kan gemanipuleerd worden als \textit{refresh} mogelijkheid \cite{lindgren2001measurement}. 
Het nadeel van \verb|<meta>| is dat gebruikt kan worden door spammers om zoekmachines voor de gek te houden \cite{Meta}. 

Ook is het mogelijk om AJAX (\textit{Asynchronous JavaScript And XML}) te gebruiken, alleen is dit complexer dan andere twee mogelijkheden, 
aangezien het een combinatie van webtechnologieën betreft \cite{singh2012ajax}.  
\begin{itemize}
    \item[\ding{226}] Op welke wijze kan de gekozen programmeermogelijkheid geïmplementeerd worden en welke functies of methoden horen daar bij
\end{itemize}
Om deze deelvraag te beantwoorden is er gekozen om \textit{JavaScript} (JS) te gebruiken. 
De andere twee mogelijkheden AJAX en \verb|<meta>| vallen af, omdat AJAX te ingewikkeld is voor dit simpele concept en \verb|<meta>| onveiliger is dan JS in verband met spammers. 

Uit het onderzoek is het duidelijk geworden dat met JS mogelijk is om een de webinterface continu \textit{up-to-date} te houden. 
In JS wordt deze methode ook wel \verb|location.reload()| genoemd. 
Als een \textit{reload} wordt gebruikt in een webserver, is het wel essentieel om een delay toe te voegen, want anders zal de webserver te veel \textit{refresh requests} krijgen. 
Een delay van 2 seconden is dan ook nodig voor de DHT11-sensor, aangezien de sensor minimaal 2 seconden nodig heeft om de data op een nauwkeurige manier te meten en beschikbaar te stellen. 

Dit kan uitgevoerd worden in JS met de volgende code \cite{freecodecamp}:
\begin{lstlisting}[language=JavaScript]
// De webpagina om de 2 seconden verversen:
setTimeout(function(){
    location.reload();
}, 2000); // 2000 millisecondes = 2 secondes
\end{lstlisting}
\begin{itemize}
    \item[\ding{226}] Onderzoek hoe en welke taal geschikt is om de realtime data te presenteren via de webinterface
\end{itemize}
Uit literatuuronderzoek \cite{Krause2016} is gebleken dat er maar één gestandaardiseerde taal wordt gebruikt voor webinterfaces en dat is HTML (\textit{Hypertext Markup Language}). 
HTML is de kern van een website en daar om heen kunnen andere talen gebruikt worden om de website een betere \textit{user experience} aan te bieden. 
Voorbeelden van die talen kunnen zijn: CSS (\textit{Cascading Style Sheets}), JS (\textit{JavaScript}), PHP (\textit{Hypertext Preprocessor}) of eventueel Python.

Uit een ander onderzoek \cite{electronicshub} is een voorbeeld (HTML) code interessant gebleken om deze aandachtig te bestuderen, zodat het geleerde toegepast kan worden in dit onderzoek. 
Door het geleerde is een manier gevonden om de gemeten data \textit{realtime} te presenteren (door gebruik van HTML/CSS/JS):
\begin{lstlisting}[language=HTML]
<!DOCTYPE html> <html>
 <head><meta charset=\"UTF-8\"><meta name=\"viewport\" content=\"width=device-width, initial-scale=1.0, user-scalable=no\">
  <title>realtime DHT11-sensor gegevens</title>
  <style>html { font-family: Helvetica; display: inline-block; margin: 0px auto; text-align: center;}
  h1 {color: #444444;margin: 50px auto 30px;}
  p {font-size: 24px;color: #888;}
  </style>
</head>
 <body>
  <h1>realtime DHT11-sensor gegevens:</h1>
   <p>Luchtvochtigheid: " + String(luchtVocht) + "%</p>
   <p>Temperatuur: " + String(tempGC) + "C""</p>
   <script>
    setTimeout(function() {
     location.reload(); 
    }, 1000);
   </script>
 </body>
</html>
\end{lstlisting}
% Bij de bovenstaande code wordt er gebruik gemaakt van de \verb|setTimeout(function(){})| methode. 
% Deze methode wordt gebruikt wanneer je andere code (tussen de brackets) wil uitvoeren met een bepaalde wachttijd (deze wachttijd staat in de parameter van \verb|function()| in milliseconden) \cite{freecodecamp2}.