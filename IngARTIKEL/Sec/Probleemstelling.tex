Een DHT-sensor gecombineerd met een MCU zou het monitoren van luchtvochtigheid en temperatuur in 
diverse toepassingsgebieden, als een simpele oplossing mogelijk kunnen maken, zoals in een huiskamer.  
Helaas ontbreekt een simpele hard- en software ontwerp om de gegevens vanuit DHT-sensor te verzamelen en op een bepaalde manier deze data 
\textit{realtime} aan een eindgebruiker beschikbaar te stellen middels een webinterface. 

% Het centrale probleem hierbij is dan ook: \\
% ``\textit{Ontwikkel een werkend systeem waarmee door de DHT11-sensor realtime gemeten luchtvochtigheid en temperatuurgegevens  
% verwerkt kunnen worden door de ESP32 MCU om deze daarna op een toegankelijke wijze te presenteren via een webinterface.}''

\subsection{Probleemanalyse}
Het probleem kan worden beschreven als het ontbreken van een oplossing voor het realtime visualiseren van de meetgegevens van DHT-sensor via een webinterface. 
Om de spelende probleem te kanaliseren, is het nodig om de technische aspecten van de te gebruiken hardware te doorgronden en te analyseren.

Een DHT-sensor is een gecombineerde luchtvochtigheid en temperatuursensor die alleen hardware-matig meet en deze gegevens beschikbaar stelt. 
Technisch gezien ontbreekt aan deze sensor dataverwerking en de communicatie mogelijkheden om de meetgegevens te kunnen presenteren via een webinterface.
% ESP32 is een krachtige SoC (\textit{System on a Chip}) 
Om dat mogelijk te maken is er een MCU als hardware nodig die de meetgegevens van de DHT-sensor uit kan lezen, verwerken en vervolgens presenteren via een webinterface. 
Een ander probleem hierbij betreft de benodigde softwareontwerp waarmee via een Wi-Fi netwerk en webinterface de meetgegevens gepresenteerd kunnen worden op een webpagina via een \textit{end-device}. 
Echter, om dit probleem op een simpele en doeltreffende manier te tackelen is het een noodzaak om de programmeermogelijkheden en de connectiviteitsopties van de hardware (DHT en MCU) te kennen. 
\subsection{Vooronderzoek}
% BRON 1:
Het doel van dit vooronderzoek is het verzamelen van technische informatie over de DHT11-sensor. 
Daarnaast zal met dit vooronderzoek een definitie keuze gemaakt worden welke MCU toegepast kan worden om het probleem op te lossen. 
Daarvoor zijn meerdere literatuur-vooronderzoeken gedaan:
\begin{itemize}
    \item[\ding{226}] 
Uit het eerste literatuuronderzoek \cite{novelan2020monitoring} is gebleken dat de DHT11-sensor in staat is om temperatuur en luchtvochtigheid te meten, en 
voorzien is van een \textit{calibrated digital signal} voor de uitgang-pen.
De DHT11-sensor maakt gebruik van de \textit{calibration coefficient} dat in het ``OTP'' (\textit{One-Time-Programmable}) programmageheugen bewaard wordt. 
Wanneer de sensor de temperatuur- en vochtigheidsveranderingen meet, leest de sensor de \textit{sensor coefficient} uit en voert deze een berekening uit. 
De DHT11-sensor wordt in klein formaat gefabriceerd, waardoor het praktisch in gebruik is.  
Ook hebben de onderzoekers van dat dezelfde artikel gebruik gemaakt van een ESP8266 MCU. 
De reden waarom ze deze MCU hebben gekozen is, omdat de ESP8266 een Wi-Fi chip bevat. 
De Wi-Fi chip is essentieel voor het opzetten van een Wi-Fi netwerk, dat gebruikers verbindingsmogelijkheden biedt om data af te lezen via een webserver. 
Daarnaast biedt deze MCU genoeg I/O-pennen, zodat die ingezet kunnen worden voor een \textit{monitoring and controlling application}. 
Om deze MCU te programmeren vertellen de auteurs dat ze gebruik hebben gemaakt van het software ontwikkel taal en programma genaamd ``Arduino''. 
\linebreak\linebreak
De auteurs hadden in hun artikel bijna dezelfde probleemstelling gedefinieerd als in dit artikel. Het betrof de volgende probleemstelling: 
``\textit{Describing the problem clearly will help in designing and making a temperature and humidity monitoring system tool}''. 
De auteurs hebben een aantal methoden toegepast, zoals literatuuronderzoek, deskresearch en hardware- en softwareontwerp. 
% VOLGENDE LIJN:
\linebreak
    \item[\ding{226}] In de tweede literatuuronderzoek \cite{srivastava2018measurement} wordt vermeld dat deze DHT11-sensor 
een complexe luchtvochtigheid en temperatuur meting bevat met een gekalibreerde digitale signaaluitvoer. 
Het betreft een gecombineerde module voor het waarnemen van vochtigheid en temperatuur. 
% In dat onderzoeksartikel is onderzocht naar de werking van de DHT11-sensor en hoe gemeten data \textit{realtime} te presenteren is op een LCD-scherm. 
%     wordt verteld dat de DHT11-sensor in staat is om 
% luchtvochtigheid- en temperatuur te meten, dat ondersteund wordt door een \textit{calibrated digital signal} uitgang-pen. 
De DHT11-sensor bestaat uit een \textit{resistive} luchtvochtigheids meetcomponent en een NTC \textit{(Negative Temperature Coefficient)} temperatuur meetcomponent. 
De DHT11-sensor werkt op seriële communicatie, dat wil zeggen dat data verzonden wordt via één-draad communicatie.  
Het voordeel van één-draad communicatie is dat het minder vermogen verbruikt in vergelijking met parallel-draad communicatie. 
Het nadeel is dat één-draad communicatie langer over doet om data van punt \textit{a} naar punt \textit{b} te verzenden \cite{patrick2002serial}. 
Het procestijd van deze sensor is ongeveer 4 milliseconden, daarnaast moet ook een initialisatie vertraging van één seconde ingesteld worden. 
De DHT11-sensor is klein van formaat, met een lage energieverbruik waarbij de signaaloverdracht circa tot 20 meter plaats kan vinden. 
Deze DHT11-sensor bevat vier I/O-pennen, namelijk: de $V_{cc}$-pen, de data-pen, de NC-pen (\textit{Not Connected}) en de $GND$-pen. 
% volgende lijn2
\linebreak
    \item[\ding{226}] De derde literatuuronderzoek \cite{mishra2017interfacing} gaat een stap verder in op de werking van de DHT11-sensor (kopje 3C van dat artikel). 
Onder dat hoofdstuk wordt uitgelegd dat de DHT11-sensor een temperatuur meetbereik heeft van 0 tot $50^\circ$C met een nauwkeurigheidsmarge van $\pm 2^\circ$C. 
Het meetbereik van de luchtvochtigheidssensor is 20 tot 90\% RH (\textit{Relative Humidity}) met een nauwkeurigheidsmarge van $\pm 5$\% RH. 
Ook wordt uitgelegd dat de DHT11-sensor een reageertijd heeft van 1 seconden, frequentie van 1Hz. 
\linebreak
% \linebreak\linebreak
% Uit deze artikelen is een specificatie tabel gemaakt van de DHT11-sensor weergegeven in tabel \ref{table:tabel2}.
% \begin{table}[h]
%     \centering
%     \label{table:tabel2}
%     \begin{tabular}{|l|l|}
%     \hline
%     \textbf{Specificaties}               & \textbf{DHT11-11 sensor}        \\ \hline
%     \textit{Communicatie Protocol}       & Seriële één-draadcomm. \\ \hline
%     \textit{Temperatuur meetbereik}      & 0 - 50 graden met $\pm 2^\circ$C      \\ \hline
%     \textit{Luchtvochtigheid meetbereik} & 20 - 90\% RH met $\pm 5$\% RH      \\ \hline
%     \textit{Voedingsbereik}              & 3 - 5.5V                      \\ \hline
%     \textit{Steekproeftrekking}          & 1 seconde                    \\ \hline
%     \textit{Arduino bibliotheken}        & DHT11 en Adafruit DHT11         \\ \hline
%     \textit{Signaaloverdracht}           & Max. afstand is 20 meter         \\ \hline
%     \end{tabular}
%     \linebreak
%     \caption{Specificaties van de DHT11-sensor}
%     \end{table}

    \item[\ding{226}] In de laatste artikel \cite{macheso2021design} is er onderzocht hoe de temperatuur en luchtvochtigheid gepresenteerd kan worden op een web-server. 
Dit onderzoeksartikel gaat in op de technische specificaties van de DHT22-sensor en ESP32 MCU. 
Er wordt uitgelegd dat de ESP32 een \textit{low-cost} en \textit{low-power} MCU is. 
Deze MCU is gebruikt, omdat het een Wi-Fi chip bevat die gebruikt kan worden om een webserver op te zetten. 
Om de webserver op te zetten wordt er gebruik gemaakt van HTML/CSS/JS en ESP32 Arduino bibliotheken, en deze software wordt ontwikkeld in de Arduino IDE omgeving. 
\linebreak 
\item[\ding{226}] Deze vier artikelen geven een aantal oplossingsrichtingen die mogelijk van toepassing kunnen zijn voor dit onderzoek, namelijk:
\begin{itemize}
    \item Het gebruik van een ESP32 MCU om een Wi-Fi netwerk op te zetten. Deze ESP32 MCU is gekozen, omdat dit een van de nieuwste Wi-Fi MCU is \cite{esp2313};
    \item Het bestuderen van de technische data specificaties van de DHT11-sensor, zodat op een juiste wijze een hardware ontwerp ontwikkeld kan worden;
    \item Het bestuderen en gebruiken van Arduino IDE en de bijbehorende bibliotheken, waarmee de standaard programmeermogelijkheden toegepast kunnen worden.
\end{itemize}
% Zie de technische specificaties van de ESP32 MCU in tabel \ref{table:tabel1}
% \begin{table}[h]
%     \centering
%     \label{table:tabel1}
%     \begin{tabular}{|l|l|}
%     \hline
%     \textbf{Technische Specificaties}    & \textbf{ESP32 MCU} \\ \hline
%     \textit{General Purpose Input Output}             & 36 pennen           \\ \hline
%     \textit{Analog to Digital Converter}              & 14 pennen           \\ \hline
%     \textit{Digital To Analog Converter}              & 2 pennen            \\ \hline
%     \textit{Flash geheugen grootte}   & 16 MB               \\ \hline
%     \textit{SRAM grootte}             & 250 KB              \\ \hline
%     \textit{Klok snelheid}    & Max. 240 MHz        \\ \hline
%     \textit{Wi-Fi frequentie} & 2.4 GHz             \\ \hline
%     \textit{\begin{tabular}[c]{@{}l@{}}Stroomverbruik in slaapstand\end{tabular}} & 2.5 $\mu$A \\ \hline
%     \end{tabular}
%     \linebreak
%     \caption{Specificaties van de ESP32 MCU}
%     \end{table}
\end{itemize}
\subsection{Vraagstelling}
De hoofdvraag is:
\begin{center}
    ``\textit{Hoe kan een werkend systeem ontwikkeld worden met een MCU die de realtime gemeten luchtvochtigheid en temperatuurgegevens door de DHT-sensor 
    kan uitlezen, verwerken en met een eigen Wi-Fi netwerk deze gegevens op een toegankelijke wijze kan presenteren via een webinterface op diverse end-devices?}''
\end{center}
% oude vraag = Ontwikkel een werkend systeem met een ESP32 MCU die realtime gemeten luchtvochtigheid en temperatuurgegevens door de DHT11-sensor 
% kan uitlezen en verwerken, waarna de ESP32 MCU met een eigen WiFi netwerk op een toegankelijke wijze de meetgegevens kan presenteren via een webinterface op diverse end-devices.
Om deze hoofdvraag te kunnen beantwoorden, zijn er drie deelvragen bedacht, namelijk:
\begin{enumerate}[label=\textit{\Alph*}.]
    \item Op welke simpele manier kunnen de meetgegevens van de DHT11-sensor realtime uitgelezen worden door een ESP32 MCU?
    \item Verken op welke elementaire wijze een Wi-Fi netwerk opgezet kan worden met de ESP32 MCU?
    \item Op welke wijze kunnen de meetgegevens continu ververst en gepresenteerd worden via een webinterface?
\end{enumerate}
Deze deelvragen zijn gericht op het verkrijgen van inzicht in de technische mogelijkheden van de te gebruiken hardware (DHT11-sensor en ESP32 MCU) en de mogelijke 
oplossingsrichtingen voor het ontwikkelen van een simpel en werkend systeem. 
Het beantwoorden van deze deelvragen zal uiteindelijk moeten leiden tot een oplossing voor het visualiseren van \textit{realtime} luchtvochtigheids- en temperatuurmetingen op een webpagina via een \textit{end-device}.