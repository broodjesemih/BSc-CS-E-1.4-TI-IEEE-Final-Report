In de discussie wordt gereflecteerd naar de werkwijze van hoe het hard- en software ontwerp heeft geleid tot een werkend systeem. 
Hierbij worden de geslaagde punten en mogelijke verbeterpunten van het onderzoek besproken. 

In dit onderzoek wordt de ESP32 MCU ingezet om een AP te creëren, waarna andere gebruikers verbinding kunnen maken met die AP, zodat de webinterface bezocht kan worden. 
Deze AP is op een elementaire manier opgebouwd, omdat andere configuratieopties niet van belang waren op de probleemstelling. 
Echter had er wel beveiligingsmodi geconfigureerd kunnen worden, zodat het netwerk minimaal beschermd is tegen ongeautoriseerde gebruikers. 

Een ander belangrijk verbeterpunt van het onderzochte ontwikkelde systeem is de toegankelijkheid van de meetgegevens via een webinterface op diverse \textit{end-devices}. 
Hoewel de webinterface minimaal functioneel was en de \textit{realtime} meetgegevens kon presenteren, had het ook uitgebreider gekund. 
Denk hierbij aan het bijhouden van de meetgegevens en deze te presenteren in een grafiek of het implementeren van waarschuwingssystemen, 
zodat gebruikers direct op de hoogte worden gesteld van eventuele waarschuwingsmetingen. 

Een van de geslaagde punten is hoe het onderzoek uitgevoerd is. 
Het onderzoeken van andere artikelen heeft een breder inzicht gegeven op het domein van de DHT11-sensor, microcontrollers en het opzetten van een webinterface. 
