Hoewel het onderzoek succesvol was in het ontwikkelen van een werkend systeem met een ESP32 MCU 
voor het uitlezen en verwerken van realtime gemeten luchtvochtigheids- en temperatuurgegevens via de DHT11-sensor, 
zijn er enkele aanbevelingen die kunnen helpen bij het uitbreiden van het systeem. 
\\
Het uitbreiden van dit concept is zeker mogelijk en aan te bevelen. 
Aanbevolen uitbreidpunten kunnen zijn:
\begin{itemize}
    \item Configuratie opties van de AP: In dit onderzoek was het essentieel om een elementaire Wi-Fi netwerk te configureren, zodat gebruikers op afstand de webinterface kunnen bezoeken. 
Daarom was gekozen om geen promiscue-modus, continu-scan-modus en beveiligingsmodi te configureren voor de AP. 
In een vervolgonderzoek zou eventueel beveiligingsmodi geïmplementeerd kunnen worden, zodat alleen toegankelijke gebruikers verbinding kunnen maken met de AP die toegang hebben de webinterface kunnen bezoeken. 

    \item AP uitbreiden met STA/AP-modus: door STA/AP-modus te configureren in plaats van alleen AP kan de ESP32 MCU zowel een client zijn van een thuisnetwerk als een eigen AP zijn. 
Gebruikers die thuis wonen hoeven dan alleen het IP-adres van de ESP32 in te vullen om de webinterface te bezoeken. 
Andere ongeautoriseerde mensen (zoals visite) kunnen verbinden met de AP en daarvandaan de webinterface bezoeken. 
Echter is het belangrijk dat een juiste beveiliging wordt ingesteld. 
    \item Het hardware-matige uitbreiden met een LCD of OLED display: Ook is deze uitbreiding aan te raden, want met een scherm is het mogelijk om de \textit{realtime} data te presenteren. 
Een pluspunt van deze uitbreiding zou zijn dat de display gemonteerd kan worden aan een muur, waarna de gemeten data direct leesbaar is. 
    \item Database: Naast de realtime gemeten data te presenteren via de webinterface, zou het helemaal mooi zijn als de gegevens worden opgeslagen in een database. 
Door een database te gebruiken kunnen gegevens uitgezet worden tegen de tijd, waarna een grafiek op te stellen is. 
\end{itemize} 