Het vastleggen van omgevingseigenschappen, zoals luchtvochtigheid en temperatuur speelt een rol in de verschillende domeinen en worden bijvoorbeeld toegepast in consumptiegoederen, testapparatuur, verwarming, ventilatie en airconditioning \cite{10}. 
Een veelgebruikt digitaal component voor het meten van luchtvochtigheid en temperatuur is de ``DHT-sensor'' (\textit{Digital Humidity Temperature} sensor) \cite{7855973}. 
Er bestaan verschillende soorten DHT-sensoren, maar in dit artikel wordt voornamelijk verwezen naar de DHT versie als het gaat om DHT11-sensor. 

In dit artikel wordt specifiek ingegaan op het ontwerpen van hard- en software met een DHT-sensor in combinatie met een MCU unit (MCU),  
die de gemeten temperatuur en luchtvochtigheid waarden \textit{realtime} kan presenteren op een \textit{end device} met behulp van een webinterface. 

Het probleem dat hier speelt betreft het continu meten van de \textit{realtime} meetwaarden van een ruimte en deze op een toegankelijke wijze te presenteren via een webinterface. 

% Hoewel de DHT-sensor een handige hulpmiddel is voor het hardwarematig verzamelen van de meetgegevens, 
% bestaat er nog steeds een technisch probleem in het visualiseren van deze gegevens dat opgelost moet worden. 
Het technische probleem kan dan vertaald worden als het overdragen van de meetgegevens aan een MCU die deze \textit{realtime} 
meetgegevens via een op te zetten webinterface meetgegevens kan visualiseren middels een \textit{end device}. 

De relevantie van het oplossen van dit technische probleem is dat het gebruikers in staat stelt om op afstand (\textit{wireless}) en \textit{realtime} de luchtvochtigheid en 
temperatuur gegevens van een ruimte afgelezen kunnen worden. 
Dit kan bijvoorbeeld toegepast worden met een IoT (\textit{Internet of Things}) toepassing in de landbouw industrie, waar nauwkeurige realtimegegevens nodig kunnen zijn voor het nemen van beslissingen, bijvoorbeeld het optimaliseren van landbouwprestaties \cite{articl2e}. 

Het doel van dit onderzoek betreft oplossen van technische probleem met een werkend systeem. 
Om dit doel te behalen is onderzoek gedaan toepassingsmogelijkheden van een MCU in combinatie met de DHT-sensor.  

In het onderzoek is literatuur- en component specificatie documentatieonderzoek als methode gekozen. 
% De methode die in dit artikel wordt toegepast is gebaseerd op het gebruik van de Arduino IDE (ontwikkelomgeving) in combinatie met de DHT-sensorbibliotheek. 

Het onderzoeken van de werking  en toepassingsmogelijkheden van de DHT-sensor, MCU en het implementeren van het visualiseren via een webinterface vormt het kern van dit onderzoek. 
De technische aspecten en werking ervan worden in dit artikel besproken. 
\subsection{Overzicht}
De lezer zal door dit artikel in de volgende volgorde worden begeleid:
\begin{itemize}
    \item[\ding{226}]\autoref{sec:probleem}: \textit{Probleemstelling} worden de hoofdvraag en deelvragen geformuleerd;
    \item[\ding{226}]\autoref{sec:methods}: \textit{Methoden} worden de toegepaste methoden uitgebreid uitgelegd;
    \item[\ding{226}]\autoref{sec:results}: \textit{Resultaten} worden de resultaten van de gebruikte methode uitgewerkt;
    \item[\ding{226}]\autoref{sec:conclusion}: \textit{Conclusies} worden conclusies getrokken uit de resultaten;
    \item[\ding{226}]\autoref{sec:discussie}: \textit{Discussie} wordt een reflectie gedaan op de opgezette onderzoek.
    \item[\ding{226}]\autoref{sec:aanbeveling}: \textit{Aanbevelingen} worden aanbevelingen gedaan voor een mogelijke vervolgonderzoek.
\end{itemize}